\documentclass[12pt]{article}

\usepackage{graphicx}       % For including images
\usepackage[letterpaper, margin=1in]{geometry}
\usepackage{amssymb}

\begin{document}

\raggedright
    \begin{titlepage}
        \centering
        \huge{\textbf{MTE 322 Project One Report}}

        \vspace{15mm}
        \Large{\textbf{Angle Grinder Power Tool}}

        \vspace{15mm}
        \Large{\textbf{Prepared By:}}
        
        \Large{Nolen Zhao}

        \Large{20932971}

    \end{titlepage}

    \section{Introduction}

    \section{Assumptions}
        The following solutions are under the assumption that the $F_{xz}$ is the maximum force that the angle grinder 
        can output. If an external force equal to and opposite of $F_{xz}$ is applied at the contact point, 
        then the angle grinder will not be able to spin. It is also assumed that the force $F_{y}$ is the normal 
        force on the angle grinder applied by the user.
    
    \section{Question One}
        The general expressions for all contact forces of the gears and bearings is found with respect to $\beta$, as 
        that is the only variable present. The value of $\beta$ will affect the output torque, and propagates through the calculations 
        for forces. This will be shown below. 
        \bigskip

        First, the output torque as a function of $\beta$ is found. The distance from the contact point to the centre of the 
        shaft is given to be $\frac{4r}{5}$ where \emph{r} is the radius. The \emph{z} component of force, perpendicular to 
        the distance from the radius, can be represented below in Equation \ref{eq:perp-fxz}. 

        \begin{equation}
            F_{xz} \cdot \cos(\beta)
        \label{eq:perp-fxz}
        \end{equation}

        $F_{xz}$ is given as $30N$, so the force at the contact point is $30N \cdot \cos(\beta)$. Thus, the torque can be found below in Equation \ref{eq:output-torque} where the 
        radius was given in the SolidWorks assembly as $55mm$.

        \begin{equation}
            \tau = F_{xz}\cos(\beta) \cdot \frac{4r}{5} = 30N \cdot \cos(\beta) \cdot \frac{4 \cdot 0.055m}{5} = 1.32 \cdot \cos(\beta)Nm
        \label{eq:output-torque}
        \end{equation}

        The rotational velocity is given as $n = 2550rpm$, and should be converted into units of $rad/s$ for calculations. The conversion is 
        shown below in Equation \ref{eq:rpm-conversion}

        \begin{equation}
            \omega = n\cdot \frac{2 \pi rad}{1 rev} \cdot \frac{1min}{60s} \rightarrow 2550rpm \cdot \frac{2 \pi rad}{1 rev} \cdot \frac{1min}{60s} = 267.035rad/s
        \label{eq:rpm-conversion}
        \end{equation}

        The equation for power and ensuing output power is found below in Equation \ref{eq:output-power}

        \begin{equation}
           \mathbb{P} = \omega \cdot \tau \rightarrow \mathbb{P}_{output} = 267.035rad/s \cdot 1.32 \cdot \cos(\beta)Nm = 
        \label{eq:output-power}
        \end{equation}
    \section{Question Two}
\end{document}
